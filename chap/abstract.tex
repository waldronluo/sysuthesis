% vim:ts=4:sw=4
%
% Copyright (c) 2008-2009 solvethis
% Copyright (c) 2010-2012 Casper Ti. Vector
% Public domain.

\begin{cabstract}
    错误检测在现代工业进程以及机器人服务,特别是无规则环境的作业中,担任越来越重要的角色。我们主要研究了悬臂卡扣组装中的错误检测技术。它在工业应用以及个人机器人中有着重要的地位。

    \indent 我们希望根据卡扣组装过程的力信号特征,用小集合的特征向量来抽象组装过程,并且据此来训练支持向量机分类法,从而在组装任务的不同阶段精确地检测组装过程中发生的错误。在我们的实验中,我们通过抽象行为特征训练了一个线性支持向量机,从而进行卡扣组装的错误检测。这个方法在早/晚期的错误检测中非常有效。在早期错误检测中,抽象程度较低的行为特征集相对于高度抽象集表现得更好,这是因为在其局部时间内颗粒度更加精细的原因。在晚期错误检测中,高度抽象集表现得更好,因为对于全局的组装动作,高度抽象集更好地表示了组装动作。
    
\end{cabstract}

\begin{eabstract}

    Failure detection plays an increasingly important role in industrial processes and robots that serve in unstructured environments. This work studies failure detection on cantilever snap assemblies, which are critical to industrial use and growing in importance for personal use. 
    
    \indent Our aim is to study whether an SVM can use a small set of features abstracted as behavior representations from the assembly force signature to accurately detect failure at different stages of the task. In this work, a linear SVM was embedded with abstract behavioral features was used to classify failure detection in cantilever snap assembly problems. The approach was useful in detecting failure both during early and late stages of the task. For early stages, low-abstraction behaviors sets performed better due to their granularity and local temporal nature. For late stage analysis, high-abstraction behaviors performed better as they capture representative and global behaviors better.

\end{eabstract}

