\chapter{Discussion and Future Work}
\section{Discussion}
%significant
Our approach successfully used a small feature set of behavior representations from the RCBHT. We noted that early failure detection is possible with low abstraction levels but not otherwise. Similarly, late failure detection was possible with higher-level of abstraction labels. With LLB label set the classifier reached 99.59\% with around 20 trials, which is comparable to \cite{masonfailure} but with a much smaller feature vector size. 

%limitation
\indent However, the limitation of this approach is fatal. When something wrong happens in the robot from the begining, we can surely detect the failure. But if the machine hand got into problem during the assembly, we could not realized at the very moment. In the lateral situation the system is lack of prevention to preserve the machine which may cause severe problems. \\

\section{Future Work}
We should consider these three improvements we could make: (1)Classify the failure cases into subsets and (2) be able to detect failure as soon as it happens.\\ 
\indent Classifying the failure cases with the force/torque signal into subsets, we would have the posterior of the diviation distribution and the corresponding diviation of specific signal. Failure cases with only one or two axis of diviation can be success classified currently but we could not effectively implement it with three axis of diviation. \\
\indent Being able to detect failure at the moment it got away from the right path is necessary because these kinds of robotic problems would cause problems. The damage would be acute at the first wrong behavior. As our limitation, the hypothesis could be made after this behavior occurs. The machine arm had been some damage when it's too late.\\ 
