\chapter{Discussion and Future Work}
\section{Discussion}
The high accuracy of using the whole process of LLB and MC indicates that there are difference between success cases and failure cases and the taxonomy, RCBHT, is able to differentiate it. However, when only the Approach stage was measured, there are little information to identify the failure. In some cases, there was only one LLB in Approach state, which began at the very start and finished in the Rotation state, for some of their axis. While little information caused the lack of diversity, too much information with noise also prevent a good hypothesis. The Primitive labels of the whole process contain too much noise of viberation and friction. For only the approach stage, though the noise disturbed the accuracy, the Classifier still give good hypothesis. For the whole, the noise is too much. \\
\indent The high accuracy of identifying failure cases proves that this approach is feasible. Gradient-driven signal processing is practical in differing signals when it's data rich. When the information is less, it's harder to give a good hypothesis. Those with such difficult, we conducted the prediction with only the Approach state, though the result is not good enough. It can divide most of the failure cases but not able to identify success cases very accurately. \\
\indent Having said so, the ongoing case early in the assembly process is able to be classified as success or failure. When the prediction can be made at the begining of Rotation state, it's more likely for us to refine lateral actions, which may prevent the damage of the machine arm or breaking the cantilever snap. \\ 
\indent However, the limitation of this approach is fatal. When something wrong happens in the robot from the begining, we can surely detect the failure. But if the machine hand got into problem during the assembly, we could not realized at the very moment. In the lateral situation the system is lack of prevention to preserve the machine which may cause severe problems. \\

\section{Future Work}
We should consider these three improvements we could make: (1)Classify the failure cases into subsets and (2) be able to detect failure as soon as it happens and (3) tried to find more general solutions.\\ 
\indent Classifying the failure cases with the force/torque signal into subsets, we would have the posterior of the diviation distribution and the corresponding diviation of specific signal. Failure cases with only one or two axis of diviation can be success classified currently but we could not effectively implement it with three axis of diviation. \\
\indent Being able to detect failure at the moment it got away from the right path is necessary because these kinds of robotic problems would cause huge problems. The damage would be acute at the first wrong behavior. If the hypothesis could be made after this behavior occurs, or the second time it began, the machine arm had been have some damage.\\ 
\indent RCBHT is specially developed for this task but when it comes to dual hand assembly, this taxonomy is not yet proved to be suitable. After all, there may be more kinds of snap assembly including but not limited in different geomatric difficulty, different machine hand pre-setting. And so, solutions should be general enough for all these configurations.
