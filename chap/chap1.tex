% vim:ts=4:sw=4
%
% Copyright (c) 2008-2009 solvethis
% Copyright (c) 2010-2012 Casper Ti. Vector
% Public domain.

\chapter{Introduction}

Uncertainty us the primary factor in robot manipulation failure. In assembly tasks, failure can lead to severe damage to both robot and tool parts. Detecting failures early in an assembly is important to avoid part's damage and increase performance. Recent work detecting assembly failures mostly focus on part assembly \cite{fullmer1989parts}, tool breakage \cite{cho2005tool} \cite{hsueh2008prediction}, and threaded fastener assembly \cite{althoefer2008automated}. No work yet has focused on snap assemblies. In our work, we have used SVMs to monitor and identify failure cases early and at later stages of the assembly. This work builds on top of our previous snap assembly \cite{2012JAR-Rojas-AutHetBotAsmbly} and incorporates an SVM monitor to the state estimation capbilities of the framework. 

\indent Prior to this work, failure identification on snap assemblies was executed using a probabilistic approach named Probabilistic Relative Change-Based Hierarchical Taxonomy (pRCBHT) \cite{2012Humanoids-Rojas-pRCBHT}. The pRCBHT adds a Bayesian filter monitor on top of the RCBHT. The latter is a taxonomy composed of increasingly abstract layers that encode robot states in intuitive ways. The bayesian filter, examines the presence of encoded behavior as well as the duration of these to produce a belief about the state. In this paper, SVMs are used instead of Bayesian filtering since SVMs provide a simpler and more intuitive approach to the classification problem. In our cases, SVMs were used in conjunction with encoded labels from three different layers in the taxonomy. Levels with higher abstraction provide high-level (global spatio-temporal) information about the task, while lower levels provide more raw (local spatio-temporal) information about the task.

\indent SVMs are currently one of the most widely used classification techniques. There is another work using Support Vector Machine for idenitying assembly failure \cite{masonfailure}, using the signals with certain time stamp to construct feature vectors. Rather than this approach, RCBHT encode the origin signal data into higher level of abstraction. Our approach takes account of labels by counting the labels and constuct a much smaller feature vector than those in \cite{masonfailure}. The accuracy of this classification approach indicates our work has comparable result with \cite{masonfailure}.

\indent There are three layers of RCBHT used in our approach, Primitive layer, Motion Composite Layer, Low-level Layer, and they are increasingly abstracted. With labels of Primitive layer, early failure detection is done in our approach. With labels of Low-level Behavior layer and Motion Composite layer, (?)late stage failure detecting is done. (? Why I should talk about two stages with all approach?) The accuracy of early stage failure detection reached 93.67\% and late stage failure detection reached 99.59\%.

\indent Other works including \cite{rodriguez2011abort} and \cite{di2013bayesian} analyze signals through the use of Hidden Markov Models (HMMs). HMMs have aided in creating generalizable methods for classification of manipulation and assembly strategies. Their methods are more general, but RCBHT was specially developed for the cantilever-snap assembly and increasingly abstracted layers will be more data-rich.

\indent Our SVM classification yielded high accuracy results. However, there are also important limits. The first is that the RCBHT appraoch does not easily generalize for other assembly or manipulation problems. The second is that the failure assembly is assumed to happen from the very begining. If something happened in between the process, we can't detect it as soon but until a state finished.

\indent In section 2, we will detail our previous work including assembly strategy and RCBHT taxonomy. In section 3, we will talk about using SVM with RCBHT to classify the failure cases. In section 4, we will give a brief review about SVM. In section 5, we will depict our experiment result and have a brief discuss about the result and limitation of the approach. In the last section, we will conclude this paper and talk about our future work.
